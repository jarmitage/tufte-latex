\documentclass[a4paper]{tufte-book}

%PhD Thesis Template for the School of Electronic Engineering and Computer Science, Queen Mary University of London. Stripped from Dan Stowell's PhD.

% Combined with Tufte-LaTeX by Jack Armitage.

% Numbered headings and full contents depth
\setcounter{secnumdepth}{1}
\setcounter{tocdepth}{1}

\hypersetup{colorlinks}% uncomment this line if you prefer colored hyperlinks (e.g., for onscreen viewing)

% TODO: Sort this mess out

%%
% Book metadata
\title[The Thesis Title]{The Thesis Title 
  \\ \noindent will go here when you know it: 
  \\ \noindent a meaningful and succinct phrase}
\author[Your name here]{Your Name Here
  \\ \noindent Submitted in partial fulfillment of the requirements of the Degree of Doctor of Philosophy
  \\ \noindent in Media and Arts Technologies
}
\publisher{School of Electronic Engineering and Computer Science
  \\ \noindent Queen Mary University of London}

% \date{2018}

%%
% If they're installed, use Bergamo and Chantilly from www.fontsite.com.
% They're clones of Bembo and Gill Sans, respectively.
%\IfFileExists{bergamo.sty}{\usepackage[osf]{bergamo}}{}% Bembo
%\IfFileExists{chantill.sty}{\usepackage{chantill}}{}% Gill Sans

%\usepackage{microtype}

%%
% Just some sample text
\usepackage{lipsum}

%%
% For nicely typeset tabular material
\usepackage{booktabs}

%%
% For graphics / images
\usepackage{graphicx}
\setkeys{Gin}{width=\linewidth,totalheight=\textheight,keepaspectratio}
\graphicspath{{graphics/}}

% The fancyvrb package lets us customize the formatting of verbatim
% environments.  We use a slightly smaller font.
\usepackage{fancyvrb}
\fvset{fontsize=\normalsize}

%%
% Prints argument within hanging parentheses (i.e., parentheses that take
% up no horizontal space).  Useful in tabular environments.
\newcommand{\hangp}[1]{\makebox[0pt][r]{(}#1\makebox[0pt][l]{)}}

%%
% Prints an asterisk that takes up no horizontal space.
% Useful in tabular environments.
\newcommand{\hangstar}{\makebox[0pt][l]{*}}

%%
% Prints a trailing space in a smart way.
\usepackage{xspace}

%%
% Some shortcuts for Tufte's book titles.  The lowercase commands will
% produce the initials of the book title in italics.  The all-caps commands
% will print out the full title of the book in italics.
\newcommand{\vdqi}{\textit{VDQI}\xspace}
\newcommand{\ei}{\textit{EI}\xspace}
\newcommand{\ve}{\textit{VE}\xspace}
\newcommand{\be}{\textit{BE}\xspace}
\newcommand{\VDQI}{\textit{The Visual Display of Quantitative Information}\xspace}
\newcommand{\EI}{\textit{Envisioning Information}\xspace}
\newcommand{\VE}{\textit{Visual Explanations}\xspace}
\newcommand{\BE}{\textit{Beautiful Evidence}\xspace}

\newcommand{\TL}{Tufte-\LaTeX\xspace}

% Prints the month name (e.g., January) and the year (e.g., 2008)
\newcommand{\monthyear}{%
  \ifcase\month\or January\or February\or March\or April\or May\or June\or
  July\or August\or September\or October\or November\or
  December\fi\space\number\year
}


% Prints an epigraph and speaker in sans serif, all-caps type.
\newcommand{\openepigraph}[2]{%
  %\sffamily\fontsize{14}{16}\selectfont
  \begin{fullwidth}
  \sffamily\large
  \begin{doublespace}
  \noindent\allcaps{#1}\\% epigraph
  \noindent\allcaps{#2}% author
  \end{doublespace}
  \end{fullwidth}
}

% Inserts a blank page
\newcommand{\blankpage}{\newpage\hbox{}\thispagestyle{empty}\newpage}

\usepackage{units}

% Typesets the font size, leading, and measure in the form of 10/12x26 pc.
\newcommand{\measure}[3]{#1/#2$\times$\unit[#3]{pc}}

% Macros for typesetting the documentation
\newcommand{\hlred}[1]{\textcolor{Maroon}{#1}}% prints in red
\newcommand{\hangleft}[1]{\makebox[0pt][r]{#1}}
\newcommand{\hairsp}{\hspace{1pt}}% hair space
\newcommand{\hquad}{\hskip0.5em\relax}% half quad space
\newcommand{\TODO}{\textcolor{red}{\bf TODO!}\xspace}
\newcommand{\na}{\quad--}% used in tables for N/A cells
\providecommand{\XeLaTeX}{X\lower.5ex\hbox{\kern-0.15em\reflectbox{E}}\kern-0.1em\LaTeX}
\newcommand{\tXeLaTeX}{\XeLaTeX\index{XeLaTeX@\protect\XeLaTeX}}
% \index{\texttt{\textbackslash xyz}@\hangleft{\texttt{\textbackslash}}\texttt{xyz}}
\newcommand{\tuftebs}{\symbol{'134}}% a backslash in tt type in OT1/T1
\newcommand{\doccmdnoindex}[2][]{\texttt{\tuftebs#2}}% command name -- adds backslash automatically (and doesn't add cmd to the index)
\newcommand{\doccmddef}[2][]{%
  \hlred{\texttt{\tuftebs#2}}\label{cmd:#2}%
  \ifthenelse{\isempty{#1}}%
    {% add the command to the index
      \index{#2 command@\protect\hangleft{\texttt{\tuftebs}}\texttt{#2}}% command name
    }%
    {% add the command and package to the index
      \index{#2 command@\protect\hangleft{\texttt{\tuftebs}}\texttt{#2} (\texttt{#1} package)}% command name
      \index{#1 package@\texttt{#1} package}\index{packages!#1@\texttt{#1}}% package name
    }%
}% command name -- adds backslash automatically
\newcommand{\doccmd}[2][]{%
  \texttt{\tuftebs#2}%
  \ifthenelse{\isempty{#1}}%
    {% add the command to the index
      \index{#2 command@\protect\hangleft{\texttt{\tuftebs}}\texttt{#2}}% command name
    }%
    {% add the command and package to the index
      \index{#2 command@\protect\hangleft{\texttt{\tuftebs}}\texttt{#2} (\texttt{#1} package)}% command name
      \index{#1 package@\texttt{#1} package}\index{packages!#1@\texttt{#1}}% package name
    }%
}% command name -- adds backslash automatically
\newcommand{\docopt}[1]{\ensuremath{\langle}\textrm{\textit{#1}}\ensuremath{\rangle}}% optional command argument
\newcommand{\docarg}[1]{\textrm{\textit{#1}}}% (required) command argument
\newenvironment{docspec}{\begin{quotation}\ttfamily\parskip0pt\parindent0pt\ignorespaces}{\end{quotation}}% command specification environment
\newcommand{\docenv}[1]{\texttt{#1}\index{#1 environment@\texttt{#1} environment}\index{environments!#1@\texttt{#1}}}% environment name
\newcommand{\docenvdef}[1]{\hlred{\texttt{#1}}\label{env:#1}\index{#1 environment@\texttt{#1} environment}\index{environments!#1@\texttt{#1}}}% environment name
\newcommand{\docpkg}[1]{\texttt{#1}\index{#1 package@\texttt{#1} package}\index{packages!#1@\texttt{#1}}}% package name
\newcommand{\doccls}[1]{\texttt{#1}}% document class name
\newcommand{\docclsopt}[1]{\texttt{#1}\index{#1 class option@\texttt{#1} class option}\index{class options!#1@\texttt{#1}}}% document class option name
\newcommand{\docclsoptdef}[1]{\hlred{\texttt{#1}}\label{clsopt:#1}\index{#1 class option@\texttt{#1} class option}\index{class options!#1@\texttt{#1}}}% document class option name defined
\newcommand{\docmsg}[2]{\bigskip\begin{fullwidth}\noindent\ttfamily#1\end{fullwidth}\medskip\par\noindent#2}
\newcommand{\docfilehook}[2]{\texttt{#1}\index{file hooks!#2}\index{#1@\texttt{#1}}}
\newcommand{\doccounter}[1]{\texttt{#1}\index{#1 counter@\texttt{#1} counter}}

% Creates an inline graphic
% https://tex.stackexchange.com/questions/139143/how-to-set-an-inline-figure-matching-the-text-height
% https://tex.stackexchange.com/questions/147642/how-to-create-new-commands-with-multiple-arguments
\newcommand{\inlinegraphic}[2]{
  \begingroup\normalfont
  \includegraphics[height=#1]{#2}%
  \endgroup
}

% NOTE: 1.5-2x line spacing required, see guidelines
% https://en.wikibooks.org/wiki/LaTeX/Paragraph_Formatting#Line_spacing
\usepackage{setspace}
\onehalfspacing

% Generates the index
\usepackage{makeidx}
\makeindex

\begin{document}

%!TEX root = ../thesis.tex

\frontmatter

\maketitle

%!TEX root = ../thesis.tex

% Workaround for creating an Abstract in a book class document, see:
% https://stackoverflow.com/questions/2737326/best-method-of-including-an-abstract-in-a-latex-book

% NOTE: official limit is 300 words, see guidelines
\chapter*{\centering Abstract}
\begin{quotation}
\noindent 

Type your abstract text here

\end{quotation}
%!TEX root = ../thesis.tex

% TODO: Update/reconcile license

% v.4 copyright page
\newpage
\begin{fullwidth}
~\vfill
\thispagestyle{empty}
\setlength{\parindent}{0pt}
\setlength{\parskip}{\baselineskip}
Copyright \copyright\ \the\year\ \thanklessauthor

\par\smallcaps{Published by \thanklesspublisher}

\par\smallcaps{tufte-latex.github.io/tufte-latex/}

\par Licensed under the Apache License, Version 2.0 (the ``License''); you may not
use this file except in compliance with the License. You may obtain a copy
of the License at \url{http://www.apache.org/licenses/LICENSE-2.0}. Unless
required by applicable law or agreed to in writing, software distributed
under the License is distributed on an \smallcaps{``AS IS'' BASIS, WITHOUT
WARRANTIES OR CONDITIONS OF ANY KIND}, either express or implied. See the
License for the specific language governing permissions and limitations
under the License.\index{license}

\par\textit{First printing, \monthyear}
\end{fullwidth}



% \chapter*{Licence}

% This work is copyright \copyright \ 2016 Your Name, and is licensed under TO BE DECIDED - CHOOSE YOUR OWN OR USE CUURENT UNIVERSITY GUIDELINES the Creative Commons Attribution-Share Alike 3.0 Unported 
% Licence. To view a copy of this licence, visit \\
% \url{http://creativecommons.org/licenses/by-sa/3.0/} \\
% or send a letter to Creative Commons, 171 Second Street, Suite 300, San Francisco, California, 94105, USA.

% \begin{center}

% %	\includegraphics [width =1cm]  {images/cc}
% %	\includegraphics [width =1cm]  {images/by}
% %	\includegraphics [width =1cm]  {images/sa}
    
% \end{center}
%!TEX root = ../thesis.tex

\chapter*{Statement of Originality}

\newthought{I, [NAME], confirm} that the research included within this thesis is my own work or that where it has been carried out in collaboration with, or supported by others, that this is duly acknowledged below and my contribution indicated. Previously published material is also acknowledged below.

I attest that I have exercised reasonable care to ensure that the work is original, and does not to the best of my knowledge break any UK law, infringe any third party's copyright or other Intellectual Property Right, or contain any confidential material.

I accept that the College has the right to use plagiarism detection software to check the electronic version of the thesis.

I confirm that this thesis has not been previously submitted for the award of a degree by this or any other university.

The copyright of this thesis rests with the author and no quotation from it or information derived from it may be published without the prior written consent of the author.

\vspace{5mm}

% Note that .gitignore contains `graphics/signature*' to prevent this from being committed
Signature: \inlinegraphic{5mm}{graphics/signature.png} 

Date: Month Day, Year

\vspace{5mm}

Details of collaboration and publications can be found in \\ Section XX. % e.g. \ref{sec:publications}.
%!TEX root = ../thesis.tex

% r.7 dedication
\cleardoublepage
~\vfill
\begin{doublespace}
\noindent\fontsize{18}{22}\selectfont\itshape
\nohyphenation
Dedicated to someone.
\end{doublespace}
\vfill
\vfill
%!TEX root = ../thesis.tex

\chapter*{Acknowledgements}

I would like to thank everyone in the \textbf{\href{http://www.eecs.qmul.ac.uk/}{C4DM}} at QMUL, who... In particular...

Acknowledge communities, thank friends, makers of tools (OpenSource and other)...

Mention your loved ones... and your pets.



% Formatting: Want the Contents to be single-spaced
\singlespacing
\tableofcontents
\onehalfspacing

\listoffigures
\listoftables

%!TEX root = ../thesis.tex

\chapter*{List of abbreviations}

\begin{table}[htbp]
\begin{center}
\begin{tabular}{ll}

ANN  	&	All Nearest Neighbours		\\
ANSI 	&	American National Standards Institute		\\
ASR  	&	Automatic Speech Recognition		\\
CART    &	Classification And Regression Trees		\\
DA    	&	Discourse Analysis		\\
DFT  	&	Discrete Fourier Transform		\\
FFT   	&	Fast Fourier Transform		\\
FM    	&	Frequency Modulation		\\
GMM 	&	Gaussian Mixture Model		\\
HCI   	&	Human-Computer Interaction		\\
HMM    	&	Hidden Markov Model		\\
NIME  	&	New Interfaces for Musical Expression		\\
NN    	&	Nearest-Neighbour		\\

\end{tabular}
\end{center}
\end{table}%


%%
% Start the main matter (normal chapters)
\mainmatter

%!TEX root = ../thesis.tex

\cleardoublepage
\chapter{Introduction}
\label{ch:intro}

%choose your own headings

\section{Motivation}



\section{Aim}



\clearpage

\section{Thesis structure}


\begin{description}
\item[Chapter \ref{ch:background}] 


\item[Chapter \ref{ch:evaluation}] 


\item[Chapter \ref{ch:conclusions}] 


\end{description}


\section{Contributions}
Contributions of this thesis are:
\begin{itemize}

\item	
Chapter \ref{ch:evaluation}:
something.

\clearpage
\section{Associated publications}
\label{sec:publications}

Portions of the work detailed in this thesis have been presented in national and international scholarly publications, as follows (journal publications highlighted in bold):

\begin{itemize}

\item	
Chapter \ref{ch:background}:
Section xxx on xxx
was published as a technical report 
%\citep{}.

\item	
Chapter xxx:
An early version of some of the work was presented at the International Conference on Thing %(DAFx) 
%\citep{}.

\item	
Chapter xxx:
Accepted for publication in the \textbf{Journal of Thing Research}
%\citep{}.

\item	
Chapter xxx:
The work presented in sections of this chapter was presented at a meeting of the Thing Network
%\citep{}.

\end{itemize}

\end{itemize}

%!TEX root = ../thesis.tex

\chapter{Background}
\label{ch:background}


\chapter{Evaluation}
\label{ch:evaluation}



%!TEX root = ../thesis.tex

\chapter{Conclusions and further work}
\label{ch:conclusions}


\section{Summary of contributions}


\section{Further work}



%%
% The back matter contains appendices, bibliographies, indices, glossaries, etc.

\backmatter

\appendix

\bibliography{sample-handout}
\bibliographystyle{thesis/refs}

\printindex

\end{document}

